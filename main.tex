%%%%%%%%%%%%%%%%%%%%%%%%%%%%%%%%%%%%%%%%%
%  My documentation report
%  Objetive: Explain what I did and how, so someone can continue with the investigation
%
% Important note:
% Chapter heading images should have a 2:1 width:height ratio,
% e.g. 920px width and 460px height.
%
%%%%%%%%%%%%%%%%%%%%%%%%%%%%%%%%%%%%%%%%%

%----------------------------------------------------------------------------------------
%	PACKAGES AND OTHER DOCUMENT CONFIGURATIONS
%----------------------------------------------------------------------------------------

\documentclass[12pt,ngerman, fleqn]{book} % Default font size and left-justified equations

\usepackage[top=3cm,bottom=3cm,left=3.2cm,right=3.2cm,headsep=10pt,letterpaper]{geometry} % Page margins

\usepackage{xcolor} % Required for specifying colors by name
\definecolor{ocre}{RGB}{52,177,201} % Define the orange color used for highlighting throughout the book

%Imojis :)
%\usepackage{coloremoji}

%Euro sign
\usepackage{german}
\usepackage{eurosym}

%figure wrap
\usepackage{wrapfig}

%Special charachters
\usepackage{pifont}

%add apendix
\usepackage[toc,page]{appendix}

% spalten listen
\usepackage{blindtext}
\usepackage{tasks}

% for figure grouping
\usepackage{subcaption}
\usepackage{graphicx}

% Font Settings
\usepackage{avant} % Use the Avantgarde font for headings
%\usepackage{times} % Use the Times font for headings
\usepackage{mathptmx} % Use the Adobe Times Roman as the default text font together with math symbols from the Symbol, Chancery and Computer Modern fonts

%\usepackage{microtype} % Slightly tweak font spacing for aesthetics
\usepackage[utf8]{inputenc} % Required for including letters with accents 
\usepackage[T1]{fontenc} % Use 8-bit encoding that has 256 glyphs

% Bibliography
%\usepackage[style=alphabetic,sorting=nyt,sortcites=true,autopunct=true,babel=hyphen,hyperref=true,abbreviate=false,backref=true,backend=biber]{biblatex}
%\addbibresource{bibliography.bib} % BibTeX bibliography file
%\defbibheading{bibempty}{}

\usepackage{csquotes}
\usepackage[style=verbose-ibid,backend=bibtex]{biblatex}
\addbibresource{bibliography.bib}

\input{structure} % Insert the commands.tex file which contains the majority of the structure behind the template

\begin{document}

%----------------------------------------------------------------------------------------
%	TITLE PAGE
%----------------------------------------------------------------------------------------

%\begingroup
%\thispagestyle{empty}
%\AddToShipoutPicture*{\put(0,0){\includegraphics[width=\paperwidth,he%ight=\paperheight]{esahubble}}} % Image background
%\centering
%\vspace*{5cm}
%\par\normalfont\fontsize{35}{35}\sffamily\selectfont
%\textbf{}\\
%{\LARGE}\par % Book title
%\vspace*{1cm}
%{\Huge }\par % Author name
%\endgroup

\begingroup
\thispagestyle{empty}
\AddToShipoutPicture*{\put(0,0){\includegraphics[scale=1.25]{}}} % Image background
\centering
\vspace*{5cm}
\par\normalfont\fontsize{35}{35}\sffamily\selectfont
\textbf{Design Thinking }{\normalsize v1} \\
{\LARGE Gifting Experience}\par % Book title
\vspace*{1cm}
{\Huge Lady Killers}\par % Author name
\endgroup

%----------------------------------------------------------------------------------------
%	COPYRIGHT PAGE
%----------------------------------------------------------------------------------------

\newpage
~\vfill
\thispagestyle{empty}

%\noindent Copyright \copyright\ 2014 Andrea Hidalgo\\ % Copyright notice

%\noindent \textsc{Geschäftspalung für Gründer am KIT}\\

%\noindent \textsc{amineafia.github.io/carbook}\\ % URL

\noindent Die Challenge \\
Im Rahmen des Seminars Design Thinking (Track 1) arbeitet die Gruppe am Thema Verbesserung des Schenkerlebnisses zu Weihnachten. Dabei werden die Teilnehmer die Methoden des Design Thinking anwenden und .... \\ % License information

\noindent \textit{Januar 2017} % Printing/edition date

%----------------------------------------------------------------------------------------
%	TABLE OF CONTENTS
%----------------------------------------------------------------------------------------

\chapterimage{content.png} % Table of contents heading image

\pagestyle{empty} % No headers

\tableofcontents % Print the table of contents itself

\tableoffigures
%\cleardoublepage % Forces the first chapter to start on an odd page so it's on the right

\pagestyle{fancy} % Print headers again

%----------------------------------------------------------------------------------------
%	CHAPTER 1
%----------------------------------------------------------------------------------------
\chapterimage{befragung.png} % Chapter heading image
\chapter{Befragungen und Ergebnisse}
Das erste Zusammenkommen fand im Rahmen des ersten Treffens des Seminars statt. Hier wurde das Thema besprochen und genau abgesteckt. Mit dem ersten Treffen wurde auch die erste Befragung durchgeführt

\section{Erstes Interview  Warum Interview/Datum}
Das erste Interview im Rahmen des Seminars fand gleichzeitig am ersten Treffen statt. Das Design Thinking sieht vor, dass das Interview im Stil einer Unterhaltung durchzuführen sei und vor allem anhand von Warum- als auch Wie-Fragen die Needs sowie Insights der Befragten zum Vorschein kommen sollen. Aus diesem Grund überlegten wir uns keine große Strategie damit keine vorgefertigte Meinung unsererseits das Interview beeinflusst.\\
In einem zweier und einem dreier Team platzierten wir uns zum einen im Einkaufszentrum am Ettlinger Tor als auch auf der Kaiserstraße Höhe Marktplatz. Die Idee dahinter war, dass im Ettlinger Tor viele Passanten unterwegs sind, welche bereits ihre Geschenke einkaufen und deshalb auf Grund der Aktualität des Themas darüber sprechen werden. Die gleiche Strategie verfolgten wir für den Standort Marktplatz \\\\---------->TODO: Beschreibung\\\\

\section{Ergebnis des ersten Interviews}
\subsection{Örtlichkeit}
Das Ettlinger Tor eignete sich nicht für das Interview. Die meisten Passanten waren sehr zielgerichtet und brachten nur wenig bis keine Zeit mit sich. Aus diesem Grund gelang es uns nur vier sehr kurze und oberflächliche Interviews in einer Stunde zu führen. Wir hatten gleichzeitig das Gefühl, dass die Atmosphäre eines Einkaufszentrum angespannt und überlaufen ist.
Am Marktplatz war unser Gefühl, dass die Atmosphäre deutlich entspannter war. Die Befragten  lehnten zum einen die Interviews seltener direkt ab und nahmen sich gleichzeitig auch mehr Zeit dafür. Dadurch konnten qualitativ bessere, als auch quantitativer mehr Interviews durchgeführt werden. Hier kamen wir auf sieben Interviews.

\subsection{Learnings}
Was wir aus dem ersten Interview gelernt haben war, dass wir für jedes einzelne Interview deutlich mehr Zeit brauchen. Wir kamen in der ersten Runde auf keine Insights und Needs, weil hierfür die Interviewzeit deutlich zu gering war. Aus diesem Grund wurde uns klar, dass die Örtlichkeit der Interviews geändert werden musste.\\

Weiterhin war die Gestaltung der Interviews wichtig. Obwohl man sich selbst keinen Rahmen setzen soll und eher ein Gespräch als eine Befragung das Ziel ist. Die Problematik der ersten Interviewrunde war, dass das Schenkerlebnis sehr spezifisch nur auf den Aspekt des Geschenke besorgen und der Bescherung reduziert wurde. So haben die Befragten unsere Fragen in diese Richtung beantwortet oder wir unbewusst den Fokus in diese Richtung gelenkt.\\

Es ist uns zudem aufgefallen, dass viele Personen auf das Wort Umfrage negativ reagieren. Das gleiche geschieht, wenn man sie fragt, ob sie kurz Zeit hätten. Dementsprechend war für uns klar, dass wir hier eine andere Einleitung wählen mussten um die Personen nicht direkt abzuschrecken.

\subsection{Das erste Treffen}
Um die nächsten Schritte zu planen hat sich die Gruppe am 24.11 direkt erneut getroffen um die weiteren Interviews zu planen und Verbesserungen vorzunehmen.

\subsubsection{Die Fragen des Interviews}
Mit dem Wissen, dass wir uns selbst stark auf die Besorgung und die Bescherung fokussiert haben, bildeten wir den kompletten Schenkprozess einmal ab um uns alle Aspekte vor Augen zu halten.\\

\begin{figure}[ht]
    \centering
      \makebox[\textwidth]{\includegraphics[width=\textwidth]{prozess-schenken}}
    \caption{Verschiedene Phasen des Schenkprozesses (Eigene Darstellung)}
    \label{fig:prozess}
\end{figure}

Durch den Überblick über den Prozess konnten wir nicht nur die Besorgung und die Bescherung als Aspekte des Schenkens zu Weihnachten ausmachen, sondern auch Aspekte wie das Verpacken der Geschenke oder die eigene Zufriedenheit mit der Auswahl. Auf dieser Basis formulierten wir eine allgemeingültige Frage, welche wir zu Beginn des Interviews stellen ohne weitere Anmerkungen zu tätigen, die die befragte Person in bestimmte Richtungen lenken würde. Die Frage lautet:

\paragraph{Wie schenken Sie zu Weihnachten?}
Der Hintergrund genau dieser Frage ist, dass die befragte Person selbst den Aspekt wählt, den sie als erstes mit dem Thema assoziiert. Anschließend ist es für uns möglich mit Warum und Wie fragen den Aspekt auszuleuchten oder bei Stillstand des Gesprächs auf den nächsten Schritt des Prozesses einzugehen. Dafür haben wir folgende Hilfsfragen formuliert:

\begin{itemize}
    \item Wann machen Sie sich zum ersten Mal Gedanken über die Weihnachtsgeschenke?
    \item Woher bekommen Sie Ihre Ideen?
    \item Was beeinflusst Ihre Entscheidung für den Kauf?
    \item Wie kommen Sie an Ihre Geschenke (Wie/Wo)?
    \item Wie und wann verpacken Sie Ihre Geschenke?
    \item Wie läuft die Bescherung bei Ihnen ab?
\end{itemize}

Falls es aus dem Gespräch selbst nicht deutlich wurde, formulierten wir weiterhin Fragen, welche in jedem Interview, ob als Teil des Gesprächs oder als einzelne Frage, gestellt werden sollten:

\begin{itemize}
    \item Wie ist Ihr Verhältnis zu Weihnachten?
    \item Wichtigkeit des Schenkens und des Beschenkt-werdens
    \item Zufriedenheit mit den eigenen Geschenken?
\end{itemize}

Natürlich stellten wir im Anhang \ref{anhang:fragebogen} weiterhin die Fragen nach dem Alter und nach Kontaktdaten für spätere Tests der Prototypen.

\subsubsection{Örtlichkeit}
Neben der Fragestellung beschäftigten wir uns zudem mit der Örtlichkeit der Befragung. Hier sind wir schnell auf die Idee gekommen, Menschen dort zu befragen wo sie bereits Zeit mitbringen. Haltestellen oder der Karlsruher Hauptbahnhof sind uns direkt als Möglichkeiten eingefallen. Zum einen erfüllt es den Aspekt, dass Menschen dort länger warten und deshalb Zeit mitbringen. Weiterhin gibt es uns auch die Möglichkeit der Observation und damit Zeit, die Befragten besser auszuwählen.

\subsubsection{Einleitung}
Als letzten Punkt der Agenda für das Treffen haben wir den Begin der Befragung besprochen. Da uns aufgefallen ist, dass viele nur ungern die Wörter Befragung oder Umfrage hören, haben wir eine neue Strategie diesbezüglich entwickelt. Zum einen wollten wir weder das Wort Befragung noch Umfrage erwähnen. Stattdessen entschieden wir, dass wir uns kurz vorstellen und den Hintergrund unserer Befragung nennen. Anschließend sind wir direkt mit unserer Leitfrage gestartet und haben damit den Befragten nicht die Möglichkeit gegeben auszuweichen.

\section{Die zweite Befragung}
Die zweite Befragung fand in der Woche nach der ersten Sitzung statt. Dabei einigten wir uns aufgrund zeitlicher Gründe auf zwei unabhängige Befragungen. Wir wählten wieder die Methode der Befragung, da wir uns erhofften, damit den kompletten Prozess des Schenkens abbilden zu können. Bei der Örtlichkeit einigten wir uns auf den Karlsruher Hauptbahnhof. Wir erwarteten, dass dort viele Passanten unterwegs sein würden, welche gleichzeitig die nötige Zeit mitbringen.

\section{Ergebnisse der zweiten Befragung}
\subsection{Örtlichkeit}
Der Hauptbahnhof war die ideale Wahl für eine Befragung. Zum einen war es uns möglich die Kandidaten auszusuchen anstatt kurzfristig die Entscheidung zu treffen ob man jemanden befragt, der gerade vorbei läuft. Durch die Übersicht über die Gleise war es leicht möglich einzuschätzen, wie viel Zeit die wartenden Personen mitbringen können.

\subsection{Learnings}
Neben der Wahl der Örtlichkeit hat sich die Erarbeitung des Prozesses als auch des Fragebogens als positiv auf die Interviews ausgewirkt. Dabei ging es nicht um die Möglichkeit den Fragebogen abzuarbeiten und damit zum kompletten Prozess Antworten zu besitzen, sondern vor allem darum, Sicherheit in das Gespräch zu bringen und auch Bereiche abzudecken, die nicht offensichtlich durch den Begriff Schenken ins Gedächtnis gerufen werden. Dadurch wurden die Gespräche flüssiger und brachten bessere Ergebnisse zu Tage.

\subsection{Key findings der Interviews}
Von der organisatorischen Seite her ist uns aufgefallen, dass Umfragen nur dann positiv von den Interviewten aufgenommen werden, wenn diese keinen störenden Effekt haben. Personen auf der Straße zu interviewen, die momentan andere Dinge zu tun haben, stehen Interview eher negativ gegenüber. Entweder sie lehnen es komplett ab oder sind in ihren Antworten kurz angebunden. Aus diesem Grund sollte man entweder Personen direkt zu Interviews einladen, die somit Zeit dafür mitbringen, oder man interviewt an Orten, an denen die Interviewten die benötigte Zeit mitbringen.\\

Inhaltlich hat das Thema Schenken ein sehr breites Spektrum an Aspekten und wird auf verschiedenste Weisen umgesetzt und angenommen. Jeder der Aspekte kann dabei Probleme auslösen. Hier kann vor das Ideenfinden und die Besorgung der Geschenke als größte Probleme angesehen werden. Gleichzeitig treten sowohl absolute Verfechter des Schenkens, als auch Gegner dessen auf. Die Gründe sind dabei ebenso vielfältig: Es gibt religiöse Menschen, welche das christliche Fest in den Vordergrund rücken, als auch Gegner, die entweder das Fest selbst oder den Kommerz hinter dem Schenken ablehnen. Dem gegenüber stehen Personen, für die das Schenken dazu gehört und im Schenken eine Freude für die Beschenkten sehen.

%----------------------------------------------------------------------------------------
%	CHAPTER 2
%----------------------------------------------------------------------------------------
\chapterimage{personas.png}
\chapter{Storytelling}
Der nächste Schritt des Design Thinking beinhaltet das Storytelling. Hierbei geht es um die Wiedergabe der Geschichten der Interviewpartner an die Gruppenmitgliedern und die anschließende Entwicklung von verschiedenen Personas und Point of Views. Dieser Schritt fand vollständig während des Seminartermins am 19. November statt.

\section{Personae}
Aufgrund der Befragung entwickelten wir zuerst zehn verschiedene Personas:

\begin{tasks}(2)
    \task Der subtile Nachfrager
    \task Der Geschenk-Profi
    \task Der persönliche Schenker
    \task Der Pragmatiker
    \task Der Bastler
    \task Der Selber-Macher-Lasser
    \task Der reiche Traumerfüller
    \task Der Religiöse
    \task Der Pflicht-Schenker
    \task Der Grinch
\end{tasks}

Bei der Einteilung der Persona gab es in der Gruppe jedoch deutlich unterschiedliche Einschätzungen. Das lag verstärkt daran, dass es verschiedene Ansichten darüber gab, welche Eigenschaften der Persona relevant sind um die zentrale Motivation zu finden, und welche vernachlässigt werden können. Schlussendlich haben wir uns jedoch auf eine Einteilung geeignet:

\begin{enumerate}
    \item Der Religiöse: Die Persona lehnt schenken an Weihnachten mit dem Hintergrund des christlichen Gedankens ab. Ihr zentrale Motivation an Weihnachten ist das Beisammensein der Familie, das Leben der christlichen Werte und das feiern des Festes im traditionellen, christlichen Sinn.
    
    \item Der Grinch: Diese Persona schenkt an Weihnachten nicht, da sie das Fest selbst nicht feiert. Dies kann dabei zwei Gründe haben. Zum einen sieht die Persona Weihnachten als ein kommerzielles Fest an, was den eigentlichen Hintergrund verloren hat. Zum anderen feiert Sie das Fest nicht, da sie entweder einer anderen oder gar keiner Religion angehört.
    
    \item Der persönliche Schenker, der Geschenk-Profi, der Bastler: In diese Gruppe fallen alle Persona, welche durch ihre hohe Motivation beim Schenken auffallen. Ihre Geschenke sind stets auf den Charakter des Beschenkten zugeschnitten. Anstatt irgendetwas zu schenken, schenken sie lieber gar nichts.

    \item Der Selber-Macher-Lasser: Eine andere Person führt den kompletten Schenkprozess für diese Persona durch. Nur das Schenken selbst wird von der Persona gemacht.
    
    \item Der subtile Nachfrager, der reiche Traumerfüller, der Pragmatiker: Die Personae fragen offen oder verdeckt was die Beschenkten haben möchten. Dabei stehen sie dem Schenken selbst eher pragmatisch gegenüber.
    
    \item Der Pflicht-Schenker: Diese Persona lehnt Weihnachten aus verschiedenen Gründen ab und schenkt nur, weil es entweder eine Tradition ist oder er anderen nicht vor den Kopf stoßen möchte.\\
\end{enumerate}

Weiterhin lassen sich diese Gruppen noch grob danach ordnen, wie sie sowohl Weihnachten als auch dem Schenken gegenüber stehen. Diese Einordnung stellt dabei besser da, auf welcher Ebene des Schenkens die Needs der Persona sind: ob sie schon das Schenken selbst ablehnen oder am Geschenkprozess etwas ändern möchten.

\begin{figure}[ht]
    \centering
      \makebox[\textwidth]{\includegraphics[width=\textwidth]{positiv}}
    \caption{Weihnachten vs. Verschenken}
    \label{fig:positiv}
\end{figure}

\section{Detaillierte Personae}
Aus diesen sechs Personae beschrieben wir zwei detaillierter. Dabei wollten wir vor allem die Insights und Needs herausspiegeln, auf die wir später die Point of View aufbauen wollen. \\\\\\

\subsection{Der Pragmatiker}

\textbf{Name & Alter:} Sven Müller, 39 Jahre\\
\textbf{Familien- und sozialer Status:} Verheiratet, Vater von 2 Söhnen\\
\textbf{Stadt:} München\\
\textbf{Beruf & Einkommen:} Arbeitet im Controlling eines mittelständigen Unternehmens, etwa \EUR{65.000}\\
\hline
\paragraph{Rolle & Position im Stakeholder-Netzwerk:}
Sven Müller ist Alleinverdiener im Haushalt, da seine Frau sich z.Zt. in Elternurlaub befindet. Er selbst sieht sich als „Brötchenverdiener“ der Familie und sieht sich deshalb auch in der Position des Familienoberhaupts. Im Beruf befindet er sich im mittleren Management. Da er aufsteigen möchte engagiert er sich intensiv im Beruf. Er engagiert sich in seinem Musikverein und ist dort, vorteilhaft durch seinen Beruf, Finanzer.\\
\hline
\paragraph{Hobbies & Identifizierungsmaßnahmen:}
Sven Müller fährt sehr gerne im Sommer mit seinem Motorrad. Aufgrund der Nähe zu den Alpen unternimmt er im Sommer an verlängerten Wochenenden öfters dort eine Tour, welche über das ganze Wochenende dauert. Als Stressausgleich und aufgrund der mangelnden Bewegung in seinem Job führt er seinen Hund abends nach der Arbeit für eine größere Runde Gassi. In der Musikkapelle seines Vorortes spielt er erstes Saxophon. Das bedeutet für ihn zum einen regelmäßiges Üben, und zum anderen die Teilnahme an Konzerten. Gleichzeitig jedoch auch ein Zeitaufwand.\\
\hline
\paragraph{Pain Points & Needs / Einstellung gegenüber ... (brand, service, problem, topic):}
Sven Müller fühlt sich als ob er an mehreren Baustellen gleichzeitig arbeiten müsste. Trotz des Elternurlaubs der Frau möchte er gleichzeitig den gleichen Standard für die Familie halten. Hinzu kommt, dass die Arbeit ihm auch für die persönliche Weiterentwicklung wichtig ist, da er Karriere machen möchte. Diesen Stress möchte er durch sein Engagement im Musikverein und durch seine Hobbys lösen, was ihm durchaus gelingt. Die Familie ist für ihn zentraler Haltepunkt seines Lebens, weshalb ihm die Zeit mit seiner Frau und den Söhnen wichtig ist. Durch die Arbeit und sein Engagement kommt das inzwischen zu kurz, weshalb eines seiner zentralen Bedürfnisse ist, mehr Zeit für die Familie zu haben.\\
Weihnachten steht er positiv gegenüber. Für ihn ist es eine Zeit des Jahres, in der er sich voll und ganz seiner Familien widmen kann. Da er jedoch sehr beschäftigt ist, möchte er bei der Organisation des Festes nicht mitwirken. Ihm sind nur ein paar wenige Rahmenbedingungen wichtig, auf die er sich vorher mit seiner Frau geeinigt hat. Das Schenken ist ein weiterer, für ihn zeitraubender, Schritt, den er nur ungern jedes Jahr aufs Neue macht, ihm jedoch sehr wichtig ist. Er schenkt nur seiner Familie und dem engsten Freundeskreis etwas, steht aber jedes Jahr vor dem Problem, dass er keine Zeit hat einzukaufen oder sich etwas zu überlegen. Aus diesem Grund fragt er entweder verdeckt oder offen nach, was die Beschenkten möchten. Er hat das Gefühl, dass dieses Verhalten den Beschenkten nicht direkt gefällt, weshalb er nach Auswegen sucht, dies anders zu machen ohne dabei viel mehr Aufwand zu investieren.\\
\hline
\paragraph{Ziele & Träume:}
Sven Müllers Ziel ist es, dass er wieder mehr Zeit für die Familie hat. Er möchte die Karrierestufe aufsteigen, hat jedoch nicht das Bedürfnis, ganz oben anzukommen. Sobald er ein ausreichendes Niveau erreicht hat, genügt ihm das. Er will an seinem Engagement im Musikverein festhalten, da ihm dies sozialen Ausgleich zu Arbeitskollegen und Familie gibt. Insgesamt wünscht er sich ein ausgeglicheneres, stressfreieres Leben.\\
\hline
\textbf{Drei Dinge, ohne die er nicht Leben kann:} Seine Familie, sein Hund und sein Saxophon \\
\textbf{Ich bin gut in:} Controlling, Saxophon\\
\textbf{Ich wünschte ich wäre gut in:} Golf \\
\textbf{Mein nächstes großes Event wird:} Mein 40. Geburtstag \\
\textbf{Mein Traumurlaub war:} Zwei Wochen in einer Ferienhütte in einem Fjord in Norwegen \\
\textbf{In meinem Kühlschrank ist immer:} Bier \\
\hline

\subsection{Der Grinch}
\textbf{Name & Alter:} Christian Feldmann, 43 Jahre\\
\textbf{Familien- und sozialer Status:} Geschieden, einen Sohn\\
\textbf{Stadt:} Stuttgart\\
\textbf{Beruf & Einkommen:} LKW Fahrer, etwa \EUR{35.000}\\
\hline
\paragraph{Rolle & Position im Stakeholder-Netzwerk:}
Christian Feldmann ist geschieden und sein Sohn, 21 Jahre, wuchs bei seiner Mutter auf. Die Beziehung zu ihm beschreibt er als abgekühlt, jedoch sehen sie sich regelmäßig. In seinem Beruf hat er nur wenig Kontakt zu seinen Kollegen. Er spielt keine zentrale Rolle im Unternehmen, ist jedoch stolz auf seinen Beruf und seine Arbeit. Als alteingesessener Fan des VfB Stuttgart trifft er sich zu den zweiwöchentlichen Heimspielen mit Gleichgesinnten im Stadion und an den anderen Wochenende zum Fußballschauen in seiner Stammkneipe (AnnehmBAR). Diese besucht er öfters, da dort seine Freundeskreis ist, mit denen er Karten spielt.\\
\hline
\paragraph{Hobbies & Identifizierungsmaßnahmen:}
Christian Feldmann ist Fan des VfB, den er als seine Liebe bezeichnen würde. Ihm ist wichtig, dass andere dies erkennen, weshalb er auch außerhalb von Spielen gelegentlich das Trikot oder den Schal trägt. Obwohl er seinen Job sehr gerne macht, ist er für ihn keine Identifizierung. Er könnte auch jeden anderen Job machen, wenn er dafür qualifiziert ist. Wichtig ist ihm das Kartenspielen, da es ihn mit seinen engsten Freunde zusammenbringt.\\
\hline
\paragraph{Pain Points & Needs / Einstellung gegenüber ... (brand, service, problem, topic):}
Christian Feldmanns mäßiges Verhältnis zu seinem Sohn bedrückt ihn sehr. Obwohl sie sich regelmäßig sehen, hat sich der Sohn in eine andere Richtung entwickelt und scheint, nach Christans Ansicht, sich sehr von ihm zu unterscheiden. Früher waren beide öfters zusammen im Stadion, doch selbst die Liebe zum VfB hat der Sohn nicht übernommen. Zu seiner Exfrau hat er nur noch ein oberflächliches Verhältnis, das sich nur auf die Gespräche über den Sohn begrenzt. Sie hat inzwischen erneuet geheiratet und ist in die Vorstadt gezogen. Aufgrund seiner gescheiterten Ehe und dem schlechten Verhältnis zum Sohn fragt sich Christian des Öfteren, warum es so gekommen ist. Er sieht sich selbst als hart arbeitenden Menschen, der sich immer um die Familie gekümmert hat. Sein größtes Bedürfnis ist es, das zerrüttete Verhältnis zu seinem Sohn wieder aufzubauen.\\
Gegenüber Weihnachten hat er eine schlechte Einstellung. Seitdem er es nicht mehr mit der Familie feiert, feiert er gar kein Weihnachten mehr. Er weiß auch nicht mehr, was er den Menschen, die er langsam nicht mehr zu kennen glaubt, schenken soll, weshalb er es komplett lässt. Für ihn ist Weihnachten deshalb nur noch eine Zeit des Jahres, mit der er nichts anzufangen weiß. Da er die Kirche ablehnt, hinterfragt er auch die Tatsache, dass dieser Tag als christlicher Feiertag ein nationaler Feiertag ist. Zudem sieht er das Verhalten aller Menschen als scheinheilig an. An Weihnachten predigen sie die humanistischen Werte des Christentums, welche sie zu allen anderen Zeiten des Jahres nicht interessieren.\\
\hline
\paragraph{Ziele & Träume:}
Christian Feldmanns Ziele sind zum einen das bessere Verhältnis zu seinem Sohn. Gleichzeitig wäre er gerne wieder in einer festen Beziehung, da er an seiner Exfrau erkennt, dass dies ein Glücksfaktor im Leben sein kann. Sein Traum ist natürlich der Aufstieg des VfB in die erste Liga\\
\hline
\textbf{Drei Dinge, ohne die er nicht Leben kann:} Sein Sohn, VfB und die AnnehmBAR\\
\textbf{Ich bin gut in:} Benokkel (schwäbisches Kartenspiel) \\
\textbf{Ich wünschte ich wäre gut in:} Ein Vater sein, Fußballspielen \\
\textbf{Mein nächstes großes Event wird:}  \\
\textbf{Mein Traumurlaub war:} Die Fahrt mit Frau und Kind in einem VW Bus durch Südeuropa \\
\textbf{In meinem Kühlschrank ist immer:} Bier \\
\hline

\subsection{Der Pflicht-Schenker}
\textbf{Name & Alter:} Peter Ohlenschläger, 31\\
\textbf{Familien- und sozialer Status:} Ledig\\
\textbf{Stadt:} Karlsruhe\\
\textbf{Beruf & Einkommen:} Wissenschaftlicher Mitarbeiter (Doktorrand) am KIT, Bereich Theoretische Physik\\
\hline
\paragraph{Rolle & Position im Stakeholder-Netzwerk:}
Peter ist Single und lebt in einer 3er WG in der Karlsruhe Oststadt. Nach seinem Studium der Physik hat er am Lehrstuhl für Physik angefangen seine Doktorarbeit zu schreiben. Er versteht sich gut mit seinen Kollegen, mit denen er jeden Tag zu Mittag isst und als einen seiner Freundeskreise ansieht. Neben diesem engagiert er sich trotz Doktorstelle weiterhin im Arbeitskreis Kunst und Kultur (AKK), das er bereits seit seinem ersten Semester tut. Sven ist kein Familienmensch. Da diese weiterhin im Dorf leben fühlt er sich ihnen gegenüber fremd, da er sich weiterentwickelt hat, während sie das nicht getan haben. Aus diesem Grund besucht er die Familie unregelmäßig, häufig nur an den großen Feiern (Geburtstag, Weihnachten, ...). Zu seinen zwei Schwestern hat er hierbei jedoch noch das beste Verhältnis.\\
\hline
\paragraph{Hobbies & Identifizierungsmaßnahmen:}
Peter liebt Knobelspiele. Aus dieser Liebe hat sich später auch das Interesse an Mathematik und Physik entwickelt. Er geht gerne auf Konzerte und Festivals, vor allem in die Richtung Hard Rock/Metal. Bandshirts, lange Haare und sein Bart sieht er als seine Erkennungszeichen an. Wenn er nicht gerade auf Konzerten anzutreffen ist, tümmelt er sich entweder am AKK oder in den verschiedenen Bars, bevorzugt Irish Pubs, Karlsruhes herum. Da er dieses Hobby bereits in seiner Kindheit hatte, spielt er jetzt noch regelmäßig Computerspiele. Inzwischen zwar deutlich reduziert, setzt er sich 1-2 Woche noch vor seinen Computer.\\
\hline
\paragraph{Pain Points & Needs / Einstellung gegenüber ... (brand, service, problem, topic):}
Obwohl Peter kein inniges Verhältnis zu seiner Familie hat, hat er nicht das dringende Bedürfnis dieses Verhältnis zu verbessern. Er hat in den Freunden und Kollegen in Karlsruhe genug sozialen Halt gefunden. Natürlich interessiert ihn weiterhin die Familie, dieses Interesse basiert jedoch auf eine Art innere Pflicht die er verspürt. Wenn er in  seiner Heimat ist, ist er schnell gelangweilt und des Öfteren genervt von den Fragen und Einstellungen seiner Eltern und Geschwister.
Diese Einstellung gegenüber seiner Familie ist ein zentraler Aspekt wenn es darum geht, dass er zu Weihnachten nach Hause geht. Am liebsten würde er einfach in Karlsruhe bleiben und die Feiertage mit Freunden verbringen oder komplett seine Ruhe haben. Weihnachten steht er aber auch in anderer Ansicht negativ gegenüber: Er mag den kommerziellen Aspekt des Festes nicht. Diese gezwungene Schenken findet er sehr nervig und steht dem sehr kritisch gegenüber. Aufgrund der Distanz zu seiner Familie fällt es ihm sehr schwer, die richtigen Geschenke zu finden. Daraus folgt oft, dass er irgendetwas schenkt. Trotz dieser Ansichten gegenüber seiner Familie und Weihnachten hat er das Bedürfnis, den Familiensegen aufrecht zu erhalten. Deshalb nimmt er trotz seiner Ablehnung jedes Jahr an Weihnachten teil und beschenkt seine Familie.\\
\hline
\paragraph{Ziele & Träume:}
Peters berufliches Ziel ist es eine wissenschaftliche Karriere zu haben. Er kann sich gut vorstellen, zukünftig als Professor zu lehren. Im gefallen die Offenheit, Neugierde und der Freigeist der Universität und der Forschung.\\
\hline
\textbf{Drei Dinge, ohne die er nicht Leben kann:} Musik, das AKK und seine Freunde \\
\textbf{Ich bin gut in:} Mathematik und logisches Verständnis, Zocken,  ehrenamtliches Engagement \\
\textbf{Ich wünschte ich wäre gut in:} Schlagzeug spielen (wäre gerne Mitglied einer Band) \\
\textbf{Mein nächstes großes Event wird:} Verteidigung der Dissertation \\
\textbf{Mein Traumurlaub war:} Die jährliche Fahrt auf das Wacken \\
\textbf{In meinem Kühlschrank ist immer:} Bier \\
\hline

%----------------------------------------------------------------------------------------
%	KAPITEL 3
%----------------------------------------------------------------------------------------
\chapterimage{ideation.png}
\chapter{Ideation}
Ideation bedeutet, aus der Persona einen Point of View zu entwickeln, der wiederum zu Ideen führt. Diese Ideen sollen als Lösung des Needs der Persona verwendet werden können. 

\section{Die Persona}
Wir haben uns Gedanken gemacht, welche Persona am offensichtlichsten einen Need besitzt, der gleichzeitig einfach zu lösen ist. Dafür haben wir das zentrale Bedürfnis der Persona herausgesucht und analysiert:

\begin{itemize}
    \item \textbf{Der Pragmatiker:} Ihm ist die Familie wichtig, weshalb er versucht das richtige Geschenk für die Person zu finden. Dabei befragt er die Personen indirekt, z.B. über Geschenklisten an den Weihnachtsmann, oder direkt. Seine größte Problematik ist die geringe Zeit, die er besitzt um die Geschenke zu besorgen.
    \item \textbf{Der Grinch:} Beim Grinch sitzt das Problem tiefer, da er Weihnachten selbst ablehnt und für ihn schenken nicht mehr in Frage kommt. Er verbindet es nur noch mit den Erinnerungen an frühere Zeiten.
    \item \textbf{Der Pflicht-Schenker:} Diese Persona hat eine negative Einstellung gegenüber Weihnachten, fühlt jedoch einen sozialen Druck gegenüber seiner Familie, an Weihnachten und dem Schenken teilzunehmen.\\\\
\end{itemize}
Es fällt sofort auf, dass der Grinch ein grundsätzliches Problem mit Weihnachten besitzt und sein Need nicht durch einen einfachen Service oder Gut zu lösen ist. Wir haben uns aus diesem Grund gegen ihn entschieden. Dem Pragmatiker fehlt wiederum nur Zeit. Da er seine Ideen bereits besitzt haben wir uns gegen ihn entschieden. Es gibt bereits zu viele Services, die diese Problematik lösen können.

Schlussendlich haben wir uns für den Pflicht-Schenker entschieden. Hier sehen wir das Potential, dass man ihm durch einen einfach Service das Geschenke-Finden und –Kaufen deutlich vereinfachen kann.

\begin{figure}[ht]
    \centering
      \makebox[\textwidth]{\includegraphics[width=0.9\textwidth]{peter2}}
    \caption{Der Pflichtschenker Peter Ohlenschläger (eigene Darstellung)}
    \label{fig:peter}
\end{figure}

\section{Base of Ideation}
Für Peter den Pflicht-Schenker entwickelten wir eine Base of Ideation, also eine Fragestellung anhand der wir den Need in die verschiedenen Aspekte aufteilen können. Die zentrale Fragestellung ist dabei:

\paragraph{How might we help Peter to find inspiration?}

Daraus resultieren die Splitterfragen:
\begin{itemize}
    \item How to find ideas?
    \item How to get the presents?
    \item How to reduce the effort?
    \item How to wrap the presents?
\end{itemize}

\section{The 5 ideas}
Als nächsten Schritt machten wir uns daran, mögliche Lösungsansätze aus diesen Splitterfragen zu generieren. Dabei dachte jeder für sich selbst nach und sammelte Vorschläge, welche anschließend mit der ganzen Gruppe besprochen und ähnliche gruppiert wurden. Aus dieser Wolke an Ideen suchte jedes Gruppenmitglied eine Idee aus. Es wurden folgende fünf Ideen gewählt:

\begin{itemize}
    \item Eine Geschenkplattform, die das Geschenke-Finden und –Einkaufen vereinfacht
    \item Eine geführte Erlebniseinkaufstour, welche das Geschenke-Finden und –Einkaufen mit Spaß verbindet (z.B. anschließende Kneipentour, Sportevents, etc.)
    \item Ein Reiseservice, der einen Fluchtort vor Weihnachten anbietet (z.B. Weihnachten auf einer Insel)
    \item Ein Wellnesscenter mit Geschenkeservice. Hier soll neben dem Wellnessprogramm die Vorführung verschiedener Geschenkideen stattfinden
    \item Ein Weihnachtseinkaufscenter, indem die verschiedenen Geschenkkategorien in einzelne Läden aufgeteilt sind und durch individuellen Service der Kunde einfach und schnell Geschenke finden und kaufen kann.\\
\end{itemize}

Diese Ideen wurden anschließend von jedem Einzelnen noch erweitert. Dabei überlegte sich jeder kurz wie er die Idee weiter ausschmücken würde. Die vollständig erweiterten Ideen sind im Anhang \ref{anhang:protos} zu finden. 

Diese fünf ausgearbeiteten Ideen wurden auf Ihre Feasibility, Desirability und Viability untersucht. Dabei sind wir auf das Ergebnis gekommen, dass die Geschenkeplattform insgesamt am besten abschneidet. Aus diesem Grund haben wir diese Idee gewählt.

\begin{table}[ht]
\caption{Einschätzung der Gruppen zu Feasability, Desirability und Viability (Eigene Darstellung)}
\begin{center}
\begin{tabular}{l*{3}{c}r}
Idee & Desirability & Feasibility & Viability & Gesamt\\
\hline
Geschenkplattform & 3  & 3 & 1 & 7\\
Geführte Erlebnis-Einkaufstour & 2  & 2 & 2 & 6 \\
Weihnachten auf einer Insel & 2  & 1 & 3 & 6\\
Wellnesszentrum mit Geschenkservice & 2 & 1 & 2,5 & 5,5\\
Weihnachts-Einkaufzentrum & 1,5 & 1 & 1 & 3,5\\
\hline
\end{tabular}
\end{center}
\label{tab:feasability}
\end{table}%

%----------------------------------------------------------------------------------------
%	CHAPTER 4
%----------------------------------------------------------------------------------------
\chapterimage{protyping.png} % Chapter heading image
\chapter{Prototyping}

\begin{figure}[ht]
    \centering
      \makebox[\textwidth]{\includegraphics[width=0.9\textwidth]{ptoto_0}}
    \caption{"Hand-made" Prototypes (Version 0.1)}
    \label{fig:peter}
\end{figure}

Nach Auswahl der groben Produktidee ging es an die konkrete Ausgestaltung in Prototypen. In einer ausführlichen Diskussionsrunde zeigte sich, dass die einzelnen Probanden doch sehr unterschiedliche Vorstellungen von dem Endprodukt hatten. Um die Variantenvielfalt zu beherrschen, versuchten wir eine Art „Morphologischen Kasten“ zu bauen, in dem jeder Eigenschaft des Prototypen die möglichen Ausprägungen zugeordnet werden und die Auswahl einer Ausprägung aus jeder Eigenschaft eine Variante ergibt. Es zeigte sich, dass die Varianten aufgrund von gegenseitigen Abhängigkeiten und Ausschlusskriterien, doch besser in einer Art Prozessablauf dargestellt werden konnten, wobei den einzelnen Prozessschritten jeweils mehrere Ausprägungen zugeordnet wurden.\\

Nach der Abbildung der Gesamtkomplexität des Entwicklungsproblems wurde in einer „Rapid-Prototyping-Runde“ von jedem einzelnen Probanden in wenigen Minuten die Grobskizzierung eines Prototypen nach den eigenen Vorstellungen als „Version 0.1“ durchgeführt. Die entstandenen, eher groben, Prototypen wurden gemeinsam weiter ausgestaltet und modifiziert, um einen möglichst großen Bereich der Kombinationsmöglichkeiten abzudecken und mit offenen Augen die Reaktionen der Endnutzer im Protypen-Test beobachten zu können, ohne Ausprägungen von vorneherein auszuschließen. Die Ergebnisse wurden digital in „Balsamiq Mockups 3“ umgesetzt.

\begin{figure}[ht]
    \centering
      \makebox[\textwidth]{\includegraphics[width=\textwidth]{geschenkplattform}}
    \caption{Prozessablauf der Geschenkplattform}
    \label{fig:peter}
\end{figure}

Die dabei entstandenen Test-Prototypen werden im Folgenden vorgestellt. Eine Großansicht der einzelnen Prototypen befindet sich im Anhang. 

\section{Test-Prototypen (Version 1.0)}
Da alle Prototypen auf dem gleichen Prozessablauf basieren, weisen diese auch die gleiche Struktur in ihrer Grundfunktion auf. Grundsätzlich wird in der jeweiligen Modalität ein Profil des zu Beschenkenden definiert. Auf Basis dieses Profils werden passende Geschenkvorschläge gemacht, welche durch ein Feedback bewertet werden können. Am Ende des Prozesses können die ausgewählten Produkte näher betrachtet und entweder online gekauft werden oder es werden lokale Kaufmöglichkeiten angezeigt.

\subsection{Swipe-App (Prototyp 1)}
\begin{figure}[ht]
    \centering
      \makebox[\textwidth]{\includegraphics[width=\textwidth]{swipe-app}}
    \caption{Swipe-App (Prototyp 1)}
    \label{fig:proto1}
\end{figure}

Bei der Swipe-App wird das Profil zunächst durch eine Steckbrief definiert, wobei spezielle Tags für Hobbies und Interessen ergänzt werden können. Anschließend wird das Profil durch „What if“-Persönlichkeitsfragen ergänzt, um das Profil weiter zu detaillieren. Die anschließenden Geschenkvorschläge können durch Swipen mit „Nope“ oder „Yep“ bewertet werden oder als „Favorit“ markiert werden, wobei die Ergebnisse in entsprechenden Listen abgespeichert werden. Aus den Listen kann das favorisierte Geschenk ausgewählt werden und in der Detailsicht letztendlich online oder vor Ort gekauft werden. 

\subsection{Chat-Bot (Prototyp 2)}
\begin{figure}[ht]
    \centering
      \makebox[\textwidth]{\includegraphics[width=\textwidth]{bot}}
    \caption{Chat-Bot (Prototyp 2)}
    \label{fig:proto2}
\end{figure}

Bei dem Chat-Bot wird zunächst analog das Profil durch einen Steckbrief erstellt. Im Unterschied ist jedoch die Grundidee, die weitere Detaillierung des Profil, sowie die Generierung der Geschenkvorschläge und deren Bewertung im Dialogstil durchzuführen. Es werden also weitere Fragen zur Person gestellt und dazu Geschenkvorschläge gemacht. Je nach Eingabe des Feedbacks werden dann weitere detaillierende Fragen gestellt und neue Vorschläge generiert. Der Vorteil liegt hierbei in der ausführlicheren Form des Feedbacks, wobei nicht nur ein Ja- oder Nein-Feedback gegeben werden kann, sondern zusätzlich Begründungen, welche die weiteren Vorschläge verbessern können. Aufgrund der schweren Programmierbarkeit des Bots, könnte ein Ansatz auch auf realen Service-Mitarbeitern basieren.

\subsection{Webseite (Prototyp 3)}
\begin{figure}[ht]
    \centering
      \makebox[\textwidth]{\includegraphics[width=\textwidth]{website}}
    \caption{Webseite (Prototyp 3)}
    \label{fig:proto3}
\end{figure}

Die Webseite findet am Computer statt und befindet sich damit in einer anderen Modalität. Zusätzlich können Profile aus sozialen Medien importiert werden. Ein weiterer Vorteil besteht in der Detaillierung der Attribute des Geschenkes durch Schieberegler je nach Preis, Design, Nützlichkeit oder Einzigartigkeit. Gerade der Preis scheint ein notwendiger Filter zu sein, um zu teure oder zu billige Geschenke auszuschließen. Im Gegensatz zur Smartphone Applikation und dem Chat-Bot, werden die Vorschläge in Listenform angezeigt. Werden in dieser Liste Vorschläge durch „X“ entfernt, rücken die unteren Vorschläge nach Oben. Positiv bewertete Vorschläge rücken an den Anfang der Liste. Durch den Button „Details“ können ebenfalls weitere Informationen eingesehen werden und die Kaufoptionen ausgewählt werden. 

\subsection{Soziales Netz (Prototyp 4)}
\begin{figure}[ht]
    \centering
      \makebox[\textwidth]{\includegraphics[width=\textwidth]{social}}
    \caption{Soziales Netz (Prototyp 4)}
    \label{fig:proto4}
\end{figure}

Das soziale Netz stellt eine Art erweiterte Form der Swipe-App (Prototyp 1) dar. Neben der Definition der Profile der Beschenkten, kann auch ein Profil des Nutzers selbst erstellt werden. Dadurch können Nutzer eine Freundesliste aufbauen und auch mit den selbst erstellten Profilen ihrer Freunde nach passenden Geschenken für diese suchen. Alternativ können sie auch mit ihrem eigenen Profil nach Produkten suchen, die ihnen gefallen könnten. Neben der Geschenkesuche, kann die App also auch als ganzjährige Shopping-App genutzt werden. Eine weitere Zusatzfunktion bietet der Wunschzettel, den jede Person für sich selbst definieren kann und der wie eine „American Wedding List“ funktioniert. Freunde können diese Wunschliste einsehen und dort Geschenke auswählen und diese für die weiteren Freunde blockieren, ohne dass der Nutzer dies sieht. Damit können Wünsche geäußert werden ohne offen darüber zu sprechen und gleichzeitig Doppelkäufe von Geschenken vermieden.\\

Per Swipe können die Vorschläge vor- und zurückgeschaltet werden. Der Auswahl-Mechanismus findet durch Anklicken des jeweiligen Profils statt, wobei der Vorschlag für jedes Profil gespeichert werden kann. Die am häufigsten ausgewählten Vorschläge werden im Feed angezeigt und von der Community up- und down-gevotet. So können im Feed die beliebtesten Geschenke über alle Profile in der Community eingesehen werden. Am unteren Bildrand kann analog zu aktuellen Apps jederzeit zwischen dem eigenen Profil, der Freundesliste, der Geschenksuche und dem Feed gewechselt werden. 

%----------------------------------------------------------------------------------------
%	CHAPTER 5
%----------------------------------------------------------------------------------------
\chapterimage{testing.png} % Chapter heading image
\chapter{Testing}
\\\\---------->TODO: Julian\\\\

\vfill
\textit{Made with \ding{170} by Lady Killers} \autocite{acknow}

\appendix
\chapterimage{anhang.png}
\chapter{Anhang}
\label{chap:appendix}

\section{Anhang 1: Fragebogen}
\label{anhang:fragebogen}
\\\\---->TODO: Formular\\\\

\section{Anhang 2: Personas}
\label{anhang:personas}
\\\\---->TODO: Welche Personae kommt hier?\\\\

\section{Anhang 3: Prototypen}
\label{anhang:protos}
Dieses Anhang enthält alle prototypen im größere Forme, damit der Leser detailierte blick auf alle Features haben kann.

\begin{figure}[ht]
    \centering
      \makebox[\textwidth]{\includegraphics[height=0.83\paperheight]{prt1}}
    \caption{Swipe-App (Prototyp 1)}
    \label{fig:prt1}
\end{figure}

\begin{figure}[ht]
    \centering
      \makebox[\textwidth]{\includegraphics[height=0.83\paperheight]{prt2}}
    \caption{Chat-Bot (Prototyp 2)}
    \label{fig:prt2}
\end{figure}

\begin{figure}[ht]
    \centering
      \makebox[\textwidth]{\includegraphics[width=0.83\paperwidth,height=0.65\paperheight]{prt3}}
    \caption{Webseite (Prototyp 3)}
    \label{fig:prt3}
\end{figure}

\begin{figure}[ht]
    \centering
      \makebox[\textwidth]{\includegraphics[height=0.8\paperheight]{prt4}}
    \caption{Soziales Netz (Prototyp 4)}
    \label{fig:prt4}
\end{figure}


\end{document}